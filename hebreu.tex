\documentclass[a5paper, 11pt]{article}
\usepackage{geometry}
\usepackage{fontspec}
\usepackage[rldocument,extrafootnotefeatures]{bidi}
\usepackage{fmultico}

\setmainfont{Tinos}
\setsansfont{Arimo}
\setmonofont{Cousine}

\setlength{\parindent}{0pt}

\newcommand{\lexi}[2]{{\raggedright\textbf{#1}\hfill\LR{#2}\par}}

\title{עברית}
\date{}

\begin{document}
\maketitle



\begin{multicols}{2}
\section*{אוצר מלים}

\subsection*{א}

\lexi{אנחנו}{nous \textit{m+f}}
\lexi{אלפן (אולפן)}{oulpan \textit{m}}
\lexi{איפה}{où}
\lexi{אני}{je \textit{m+f}}
\lexi{את}{tu \textit{f}}
\lexi{אתה}{tu \textit{m}}
\lexi{אתם}{vous \textit{m}}
\lexi{אתן}{vous \textit{f}}

\subsection*{ב}

\lexi{ב}{à, dans}

\subsection*{ג}

\lexi{גר, גרה}{habiter \textit{participe}}
\lexi{גם}{aussi}

\subsection*{ה}

\lexi{ה}{le, la, les}
\lexi{הוא}{il}
\lexi{היא}{elle}
\lexi{הם}{ils}

\subsection*{ו}

\lexi{ו}{et}

\subsection*{כ}

\lexi{כן}{oui}

\subsection*{ל}

\lexi{לא}{non}
\lexi{לומד, לומדת}{étudier, apprendre \textit{participe}}

\subsection*{מ}

\lexi{מה}{quoi, qu'est-ce que ?}
\lexi{מורה, םורה}{professeur, enseignant}
\lexi{מתוס}{avion \textit{m}}
\lexi{מי}{qui ?}

\subsection*{ע}

\lexi{עהרית}{l'hébreu}

\subsection*{ש}

\lexi{שלום}{bonjour, salut (littéralement paix)}

\subsection*{ת}

\lexi{תלמיד, תלמידה}{étudiant(e), élève}



\end{multicols}
\end{document}
